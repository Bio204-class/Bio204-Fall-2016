\documentclass{beamer}

\input{../slidespreamble.tex}

%===========================================================
% Title Info
\title{Bio 204: Biological Data Analysis}
\subtitle{Introduction to R} % (optional)

\author[P. Magwene]{Instructor: Paul M. Magwene}

\date{Fall 2016}

\begin{document}
%===========================================================
\begin{frame}
\titlepage
\end{frame}

%===========================================================
\begin{frame}
  \frametitle{Overview of Lecture}

\begin{itemize}
		\item Introduction to R
		\begin{itemize}
			\item R resources
			\item Important programming concepts
			\item Introduction to data types and data structures in R 
		\end{itemize}
        \item Literate programming
		\item Hands-On Session
\end{itemize}

\end{frame}



%===========================================================
\begin{frame}
  \frametitle{What is R?}
\begin{itemize}

\item `A language and environment for statistical computing and graphics'
\item First developed in the mid-90s
\item Derives from the S language
\begin{itemize}
	\item S was developed at Bell Labs in the mid-80s
\end{itemize}

\item Advantages

	\begin{itemize}
	\item Free and open-source
	\item Much of the academic statistical community has adopted it
	\item Active developer and user community
	\item Wealth of built-in and user contributed libraries available for all types of analyses
	\end{itemize}

\item Disadvantages
	\begin{itemize}
		\item GUI not as well developed as commercial statistical packages
			\begin{itemize}
				\item S-Plus; site licensed by Duke - see OIT website
			\end{itemize}
		\item Has higher learning curve than some other simpler statistical software
		\item Command-line can be intimidating
	\end{itemize}

\end{itemize}

\end{frame}
%===========================================================

%===========================================================
\begin{frame}
  \frametitle{R Resources on the Web}
\begin{itemize}

\item Home Page
\begin{itemize}
	\item \texttt{http://www.r-project.org}
\end{itemize}

\item Comprehensive R Archive Network (CRAN)
\begin{itemize}
	\item \texttt{http://cran.r-project.org/mirrors.html}
	\item See especially the `Task Views'
			\begin{itemize}
				\item Statistical and population genetics
				\item Environmental and ecological analysis
				\item Spatial statistics
			\end{itemize}
\end{itemize}

\item Introductions and Tutorials
\begin{itemize}
	\item see \texttt{http://cran.r-project.org/other-docs.html}
\end{itemize}

\end{itemize}
\end{frame}
%===========================================================

%===========================================================
\begin{frame}

\begin{center}
\LARGE{Programming Concepts \& Vocabulary}

\end{center}
\end{frame}
%===========================================================


%===========================================================
\begin{frame}
  \frametitle{Some Important Programming Concepts}
\begin{itemize}

\item Data Types
\begin{itemize}
	\item refer to the types of values that can be represented in a computer program
	\item determine the representation of values in memory
	\item determine the operations you can perform on those values
	\item Examples: integers, strings, floating point values
\end{itemize}

\item Data Structures
\begin{itemize}
	\item a way of storing and accessing collections of data
	\item different structures are more efficient for particular types of operations
	\item Examples: lists, hash tables, stacks, queues, trees
\end{itemize}

\item Variables
\begin{itemize}
	\item Variables are references to objects/values in memory
	\item Think of them as labels that point to particular places in a computer's memory
\end{itemize}


\end{itemize}

\end{frame}
%===========================================================

%===========================================================
\begin{frame}[fragile]
  \frametitle{More Important Programming Concepts}
\begin{itemize}

\item Statement
\begin{itemize}
	\item an instruction that a computer program can execute
	\item Example: \verb=print("Hello, World!")=
\end{itemize}

\item Operators
\begin{itemize}
	\item Symbols representing specific computations
	\item Example: \verb!+, -, *! (addition, subtraction, multiplication)
\end{itemize}


\item Expression
\begin{itemize}
	\item a combination of values, variables, and operators
	\item Example: \texttt{1 + 1}
\end{itemize}


\item Functions (subroutines, procedures, methods)
\begin{itemize}
	\item A piece of code that carries out a specific task, set of instructions, calculations, etc.
	\item Typically used to encapsulate algorithms
\end{itemize}

\end{itemize}

\end{frame}
%===========================================================

%===========================================================
\begin{frame}

\begin{center}
\LARGE{Basic Data Types, Data Structures \\ and Operators in R}

\end{center}
\end{frame}
%===========================================================


%===========================================================
\begin{frame}[fragile]
  \frametitle{Arithmetic Operators and Mathematical Functions in R}


\begin{Rcode}
> 10 + 2 # addition
[1] 12
> 10 - 2 # subtraction
[1] 8
> 10 * 2 # multiplication
[1] 20
> 10 / 2 # division
[1] 5
> 10 ^ 2 # exponentiation
[1] 100
> 10 ** 2 # alternate exponentiation
[1] 100
> sqrt(10) # square root
[1] 3.162278
> 10 ^ 0.5 # same as square root
[1] 3.162278
> pi*(3)**2  # R knows some useful constants
[1] 28.27433
> exp(1) # exponential function
[1] 2.718282
\end{Rcode}

\end{frame}
%===========================================================


%===========================================================
\begin{frame}[fragile]
  \frametitle{Numeric Data Types in R}
\begin{itemize}

\item Floating point values (`doubles')
\begin{Rcode}
> typeof(10.0)
[1] "double"
\end{Rcode}


\item Integers
\begin{itemize}
  \item Default numeric type is double, must explicitly ask for integers if single values
\end{itemize}
\begin{Rcode}
> typeof(as.integer(10))
[1] "integer"
> typeof(10L) # appending L is alternate way to get an integer
[1] "integer"
\end{Rcode}

\item Complex numbers
\begin{Rcode}
> typeof(1 + 1i)
[1] "complex"
\end{Rcode}

\end{itemize}

\end{frame}
%===========================================================

%===========================================================
\begin{frame}[fragile]
  \frametitle{Boolean Values and Operators and Comparison Operators }

 Boolean (`logical') values represent binary True/False states.

\begin{Rcode}
> x <- TRUE
> typeof(x)
[1] "logical"
> y <- FALSE
> x <- T # shorthand for TRUE
> y <- F # shorthand for FALSE
\end{Rcode}

\begin{Rcode}
> TRUE & FALSE   # logical and
[1] FALSE
> TRUE | FALSE   # logical or
[1] TRUE
> !True          # logical negation
[1] FALSE
> !FALSE
[1] TRUE
> y <- FALSE
> isTRUE(y)
[1] FALSE
\end{Rcode}

\end{frame}
%===========================================================


%===========================================================
\begin{frame}[fragile]
  \frametitle{Comparison Operators}

The common comparison operators when applied to numerical values return Boolean reults.

\begin{Rcode}
> 5 < 10  # less than
[1] TRUE
> 5 > 10 # greater than
[1] FALSE
> 10 <= (5 + 5) # less than or equal to
[1] TRUE
> 10 >= (5 + 5.1) # greater than or equal to
[1] FALSE
> 10 == (2 * 5)  # equality (not be be confused with =)
[1] TRUE
> 10 != (3 + 3 + 3 + 3) # not equal to
[1] TRUE
\end{Rcode}
\end{frame}
%===========================================================


%===========================================================
\begin{frame}[fragile]
  \frametitle{Characters (strings) in R}


Character strings are written between single or double quotes.

\begin{Rcode}
> x <- 'Hello' # or x <- "Hello"
> typeof(x)
[1] "character"
> y < - "World"
> paste(x,y)  # join strings
[1] "Hello World"
> z <- paste(x,y)
> substr(z, 1, 3)
[1] "Hel"
> substr(z, 4, 8)
[1] "lo Wo"
\end{Rcode}



\end{frame}
%===========================================================



%===========================================================
\begin{frame}[fragile]
  \frametitle{Simple Data Structures in R: Vectors}

Vectors are the simplest data structure in R
\begin{itemize}
	\item vectors represent an ordered list of items
\begin{Rcode}
> x <- c(2,4,6,8)
> y <- c('joe','bob','fred')
\end{Rcode}

	\item vectors have length (possibly zero) and type
\begin{Rcode}
> typeof(x)
[1] "double"
> length(x)
[1] 4
> typeof(y)
[1] "character"
\end{Rcode}


\end{itemize}

\end{frame}
%===========================================================

%===========================================================
\begin{frame}[fragile]
  \frametitle{Simple Data Structures in R: Vectors}

Accesing the objects in a vector is accomplished by `indexing':
\begin{itemize}
	\item The elements of the vector are assigned indices $1 \ldots n$ where $n$ is the length of the vector

\begin{Rcode}
> x <- c(2,4,6,8)
> length(x)
[1] 4
> x[1]
[1] 2
> x[2]
[1] 4
> x[3]
[1] 6
> x[4]
[1] 8
\end{Rcode}


\end{itemize}

\end{frame}
%===========================================================

%===========================================================
\begin{frame}[fragile]
  \frametitle{Simple Data Structures in R: Vectors}

\begin{itemize}

\item Single objects are usually represented by vectors as well
\begin{Rcode}
> x <- 10.0
> length(x)
[1] 1
> x[1]
[1] 10
\end{Rcode}


\item Every element in a vector is of the same type

\begin{itemize}
	\item If this is not the case the the values are coerced to enforce this rule
\end{itemize}

\begin{Rcode}
> x <- c(1+1i, 2+1i, 'Fred', 10)
> x
[1] "1+1i" "2+1i" "Fred" "10"
\end{Rcode}

\end{itemize}

\end{frame}
%===========================================================


%===========================================================
\begin{frame}[fragile]
  \frametitle{Arithmetic Operators Work on Vectors in R}

Most arithmetic operators and mathematical functions work element-by-element on vectors in R

\begin{Rcode}
> x <- c(2, 4, 6, 8)
> y <- c(0, 1, 2, 3)
> x + y
[1]  2  5  8 11
> x - y
[1] 2 3 4 5
> x * y
[1]  0  4 12 24
> x^2
[1]  4 16 36 64
> sqrt(x)
[1] 1.414214 2.000000 2.449490 2.828427
\end{Rcode}

\end{frame}
%===========================================================



%===========================================================
\begin{frame}
  \frametitle{Things to Remember}

\begin{itemize}

 \item Try it out - programming involves experimentation
 \item Practice - learning to program, like learning a foreign language, requires lots of practice.
 \item Persist - many new tools/concepts can be hard to grasp at first. Keep plugging away until you get that `Aha!' moment
\end{itemize}


\end{frame}
%===========================================================




\end{document}
